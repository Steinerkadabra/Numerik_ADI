\documentclass[a4aper,pagesize]{scrartcl}
\usepackage[utf8]{inputenc}
\usepackage[T1]{fontenc}
\usepackage[ngerman]{babel}
\usepackage{enumitem}
\usepackage{amsmath}
\usepackage{amssymb}
\usepackage{amsthm}
\usepackage{breqn}
\usepackage{graphicx}
\usepackage{pdfpages}
\usepackage{float}
\usepackage{subfigure}
\usepackage{subfigure}
\usepackage{stmaryrd}
\usepackage{tikz}
\usetikzlibrary{decorations.pathreplacing}
\usepackage{wrapfig}
\usepackage{blindtext}

\theoremstyle{definition}
\newtheorem{mydef}{Definition}[section]
\theoremstyle{plain}
\newtheorem{thm}[mydef]{Theorem}
\theoremstyle{remark}
\newtheorem{bem}{Bemerkung}[section]

\newcommand{\func}[3]{$ #1 : #2 \rightarrow #3 $}
\newcommand{\xvecn}[1]{$( #1 _1,...,#1 _n)$}
\newcommand{\xser}[2]{$ #1 _1,...,#1 _#2$}
\newcommand{\ser}[2]{ #1 _1,...,#1 _#2}
\newcommand{\xvec}[2][n]{$( #2 _1,...,#2 _{#1})$}
\newcommand{\RR}{\mathbb{R}}
\newcommand{\NN}{\mathbb{N}}
\newcommand{\PP}{\mathbb{P}}
\newcommand{\EE}{\mathbb{E}}
\newcommand{\cO}{\mathcal{O}}
\newcommand{\cA}{\mathcal{A}}
\newcommand{\cB}{\mathcal{B}}
\newcommand{\cF}{\mathcal{F}}
\newcommand{\Borel}{\mathcal{B}}
\newcommand{\bernulli}{\mathrm{Ber}}
\newcommand{\uniform}{\mathrm{U}}
\newcommand{\indicator}{\mathbbm{1}}
\newcommand{\pderiv}[2]{\frac{\partial #1}{\partial #2}}
\newcommand{\di}{\mathrm{d}}
\renewcommand{\hat}{\widehat}

%setzt den hervorhebungsstil
\DeclareTextFontCommand{\emph}{\bfseries}

%verhindert einrückung bei neuem absatz
%\setlength{\parindent}{0em} 

\title{Title}
\date{\today}
\author{Thomas Steindl, Daylen Thimm}

\begin{document}

\maketitle

\section{Van Neumann Stability Analysis: Explicit Scheme}
The explicit scheme for the two dimensional heat equation is
\begin{align}
	&\frac{T_{j,k}^{n+1} - T_{j,k}^{n}}{\Delta t}
	= \frac{T_{j-1,k}^{n} - 2 T_{j,k}^{n} + T_{j+1,k}^{n}}{(\Delta x)^2}
	+ \frac{T_{j,k-1}^{n} - 2 T_{j,k}^{n} + T_{j,k+1}^{n}}{(\Delta y)^2}
\Leftrightarrow\\
\Leftrightarrow
	&T_{j,k}^{n+1}
	= T_{j,k}^{n}
	+ \frac{1}{\rho} \left(
		T_{j-1,k}^{n}
		+ T_{j+1,k}^{n}
		+ T_{j,k-1}^{n}
		+ T_{j,k+1}^{n}
		- 4 T_{j,k}^{n}
	\right)
\end{align}
assuming $\Delta x = \Delta y =: h$ and setting $\rho = \frac{h^2}{\Delta t}$. We begin our Van Neumann stability analysis by inserting the fourier series
\begin{equation}
	T_{j,k}^n = \sum_{p,q = 0}^{N-1} \hat{T}^n_{p,q} e^{ipx_j + iqy_k}
\end{equation}
into the numerical scheme. We obtain
\begin{equation}
	\begin{split}
		\sum_{p,q = 0}^{N-1} \hat{T}^{n+1}_{p,q} e^{i(px_{j} + qy_{k})} =
		\sum_{p,q = 0}^{N-1} (
			&\hat{T}^{n}_{p,q} e^{i(px_{j} + qy_{k})} +
			\frac{1}{\rho} (
			  	  \hat{T}^{n}_{p,q} e^{i(px_{j-1} qy_{k})}
				+ \hat{T}^{n}_{p,q} e^{i(px_{j+1} + qy_{k})} + \\
			   &+ \hat{T}^{n}_{p,q} e^{i(px_{j} + qy_{k-1})}
				+ \hat{T}^{n}_{p,q} e^{i(px_{j} + qy_{k+1})}
				-  4 \hat{T}^{n}_{p,q} e^{i(px_{j} + qy_{k})}
			)
		).
	\end{split}
\end{equation}
Using that $x_{j+1} = {x_j} + h$ and $x_{j-1} = {x_j} - h$ and that the discretization in $x$ and $y$ are the same we get
\begin{equation}
	\sum_{p,q = 0}^{N-1}(
		  \hat{T}^{n+1}_{p,q}
		- \hat{T}^{n}_{p,q}
		-\frac{1}{\rho} (
			  \hat{T}^{n}_{p,q} e^{-iph}
			+ \hat{T}^{n}_{p,q} e^{iph}
		    + \hat{T}^{n}_{p,q} e^{-iqh}
			+ \hat{T}^{n}_{p,q} e^{iqh}
			-  4 \hat{T}^{n}_{p,q}
		)
	)\, e^{i(px_{j} + qy_{k})} = 0
\end{equation}
and because $(e^{i(px_{j} + qy_{k})})_{j,k \in \{0, ..., N-1\}}$ is a basis of the triginometrical polynomials of degree $N$ in two variables we have for all $p,q \in \{0, \dots, N-1\}$ that
\begin{align*}
	  &\hat{T}^{n+1}_{p,q}
	- \hat{T}^{n}_{p,q}
	-\frac{1}{\rho} \left(
		  \hat{T}^{n}_{p,q} e^{-iph}
		+ \hat{T}^{n}_{p,q} e^{iph}
	    + \hat{T}^{n}_{p,q} e^{-iqh}
		+ \hat{T}^{n}_{p,q} e^{iqh}
		-  4 \hat{T}^{n}_{p,q}
	\right)= 0
&\Leftrightarrow\\
\Leftrightarrow
	  &\hat{T}^{n+1}_{p,q} =
      \hat{T}^{n}_{p,q}
	 +\frac{1}{\rho} \left(
		  \hat{T}^{n}_{p,q} e^{-iph}
		+ \hat{T}^{n}_{p,q} e^{iph}
	    + \hat{T}^{n}_{p,q} e^{-iqh}
		+ \hat{T}^{n}_{p,q} e^{iqh}
		-  4 \hat{T}^{n}_{p,q}
	\right)
&\Leftrightarrow\\
\Leftrightarrow
	  &\hat{T}^{n+1}_{p,q} =
      \hat{T}^{n}_{p,q}\left(
      1
	 +\frac{1}{\rho} \left(
		  e^{-iph}
		+ e^{iph}
	    + e^{-qh}
		+ e^{qh}
		-  4
	    \right)
	\right)
&\Leftrightarrow\\
\Leftrightarrow
	  &\hat{T}^{n+1}_{p,q} =
      \hat{T}^{n}_{p,q}\left(
      1
	 +\frac{1}{\rho} \left(
		  2\cos(ph)
	    + 2\cos(qh)
	    - 4
	    \right)
	\right)
&\Leftrightarrow\\
\Leftrightarrow
	  &\hat{T}^{n+1}_{p,q} =
      \hat{T}^{n}_{p,q}\underbrace{\left(
      1
	 -\frac{4}{\rho} \left(
	      \sin^2\left(\frac{ph}{2}\right)
	    + \sin^2\left(\frac{qh}{2}\right)
	    \right)
	  \right)}_{=:G_{p,q}},
\end{align*}
where in the last step the identity $\sin^2(\alpha/2) = \frac{1}{2}(1-\cos(\alpha))$ was used. For van Neumann stability we need that $|G_{p,q}|<1$. We therefor want to find conditions on $\Delta t$ and $h$ such that
\begin{equation}
-1 \le
	1-\frac{4}{\rho} \left(
		  \sin^2\left(\frac{ph}{2}\right)
	    + \sin^2\left(\frac{qh}{2}\right)
	\right)
	\le 1.
\end{equation}
The rhight hand side inequality is trivial and for the left hand side inequality we need that $\rho \geq 4$. This gives us the CFL condition
\begin{equation}
	\frac{h^2}{\Delta t} = \rho \geq 4 \Leftrightarrow \Delta t \leq \frac{h^2}{4} = \frac{1}{4N^2}.
\end{equation}
We conclude that $4N^2t$ time steps are needed to integrate up to a time $t$.


\section{Van Neumann Stability Analysis: Implicit Scheme}


The implicit scheme for the two dimensional heat equation is
\begin{align}
	&\frac{T_{j,k}^{n+1} - T_{j,k}^{n}}{\Delta t}
	= \frac{T_{j-1,k}^{n+1} - 2 T_{j,k}^{n+1} + T_{j+1,k}^{n+1}}{(\Delta x)^2}
	+ \frac{T_{j,k-1}^{n+1} - 2 T_{j,k}^{n+1} + T_{j,k+1}^{n+1}}{(\Delta y)^2}
\Leftrightarrow\\
\Leftrightarrow
	&T_{j,k}^{n+1}
	= T_{j,k}^{n}
	+ \frac{1}{\rho} \left(
		T_{j-1,k}^{n+1}
		+ T_{j+1,k}^{n+1}
		+ T_{j,k-1}^{n+1}
		+ T_{j,k+1}^{n+1}
		- 4 T_{j,k}^{n+1}
	\right)
\end{align}
assuming $\Delta x = \Delta y =: h$ and setting $\rho = \frac{h^2}{\Delta t}$. Again, we begin our Van Neumann stability analysis by inserting the fourier series
\begin{equation}
	T_{j,k}^n = \sum_{p,q = 0}^{N-1} \hat{T}^n_{p,q} e^{ipx_j + iqy_k}
\end{equation}
into the numerical stencil. We bring all terms to one side of the equation and again by using $x_{j+1} = {x_j} + h$ and $x_{j-1} = {x_j} - h$ and that the discretization in $x$ and $y$ are the same and also by placing the exponential outside of the brackets we get
\begin{dmath}
	\sum_{p,q = 0}^{N-1} \left(
		\hat{T}_{p,q}^{n+1}
		-\hat{T}_{p,q}^{n}
		-\frac{1}{\rho}\left(
			\hat{T}_{p,q}^{n+1} e^{-iph}
			+ \hat{T}_{p,q}^{n+1} e^{iph}
			+ \hat{T}_{p,q}^{n+1} e^{-iqh}+\\
			+ \hat{T}_{p,q}^{n+1} e^{iph}
			+ \hat{T}_{p,q}^{n+1}
		\right)
	\right)
	e^{i(px_j + qy_k)}
	=
	0.
\end{dmath}




\end{document}

